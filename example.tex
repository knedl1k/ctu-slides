\documentclass[aspectratio=169, 9pt, t, notheorems, mathserif]{beamer}

\usepackage{polyglossia}
\setdefaultlanguage{english}

\makeatletter
\def\input@path{{../configs/}{./}} 
\makeatother

\usepackage{ctu-slides}

\usepackage{fontspec}
\usefonttheme{professionalfonts} 
\setsansfont{Technika}[
    Path = fonts/,
    Extension = .otf,
    UprightFont = *-Book,
    BoldFont = *-Bold,
    ItalicFont = *-Italic,
    BoldItalicFont = *-BoldItalic
]
\usepackage{pdfpages}
\usepackage{tikz}
\usetikzlibrary{arrows.meta, calc, positioning, shapes.geometric, angles, quotes, backgrounds}

\definecolor{academicBlue}{RGB}{0, 85, 164}
\definecolor{academicRed}{RGB}{200, 30, 40}
\definecolor{marginGray}{RGB}{220, 220, 220}

\title{Automated Recognition and \\ Classification of Trading Cards} 
\subtitle{Presentation of the Unassisted project} 
\author{Jakub Adamec} 
\institute{     
    \textbf{Czech Technical University in Prague}\\
    Faculty of Electrical Engineering\\
    Department of Cybernetics\\
    Visual Recognition Group
}
\date{\today}

\begin{document}

\begin{frame}[plain, noframenumbering]
    \vspace{2cm} 
    \begin{columns}[t] 
        \begin{column}{0.65\textwidth}
            {\usebeamerfont{title}\usebeamercolor[fg]{title}\huge \textbf{\inserttitle}}\par
            \vspace{0.3cm}
            {\large \color{CVUTGrey}\insertsubtitle}\par
            
            \vspace{1.5cm}
            \textbf{\insertauthor}\par
            \vspace{0.2cm}
            \footnotesize \insertinstitute\par
        \end{column}

        \begin{column}{0.3\textwidth}
            \raggedleft
            \vspace{-0.5cm}

            \includegraphics[width=2cm, keepaspectratio]{./media/logo_CTU_vertical.pdf} 
        \end{column}
    \end{columns}
\end{frame}

\begin{frame}{Motivation and Objective}
    \textbf{The problem}
    \begin{itemize}
        \item Manual sorting of trading cards (TCG) is time-consuming and prone to errors.
        \item Existing commercial solutions (e.g., CardBot) are proprietary and/or expensive. 
        \item Many different similar cards. MTG has >100k unique cards.
    \end{itemize}

    \vspace{0.5cm}
    \textbf{Objectives of the work}
    \begin{itemize}
        \item Software: Recognition pipeline that recognizes cards even from poor-quality photos.
        \item Hardware: Affordable and widely available components (Arduino, stepper motors...).
        \item The entire project FOSS and well documented so that it can serve as educational material.
    \end{itemize}
\end{frame}

\begin{frame}{Hardware: Concept}
    \begin{itemize}
        \item Focus on affordable components.
        \item Vacuum suction cups, stepper motor, pneumatic valve, hall encoder.
        \item \textbf{Milestone reached:} Precise stepper motor control using rotation encoders feedback loop.
    \end{itemize}
    \begin{center}
        \includegraphics[width=0.58\textwidth]{./media/robotic-arm_cards.pdf}
    \end{center}
\end{frame}

\begin{frame}{Data Acquisition \& Augmentation}
    \begin{columns}[t]
        \begin{column}{0.5\textwidth}
            \textbf{Synthetic Dataset Generation}
            \begin{itemize}
                \item Real-world labeled dataset is non-existent.
                \item Scans retrieved via \textit{Scryfall API}.
                \item Custom naming convention for human-readability \& uniqueness:
            \end{itemize}
        \end{column}
        \begin{column}{0.5\textwidth}
            \textbf{Bridge the Domain Gap}
            \begin{itemize}
                \item Heavy augmentation pipeline required to match camera output.
                \item \textbf{Geometric:} Random perspective, rotation.
                \item \textbf{Photometric:} Blur, noise, lighting conditions, partial occlusions.
            \end{itemize}
        \end{column}
    \end{columns}
    \vspace{1.5em}
    \begin{center}
        \Large
        \texttt{filename = edition\_collectorNumber\_name-UUID.png}
        \texttt{zen\_21\_kor-outfitter-00006596-1166-4a79-8443-ca9f82e6db4e.png}
    \end{center}
    \vspace*{-2em}
    \begin{figure}
        \includegraphics[scale=0.7]{./media/aug.jpg}
    \end{figure}
\end{frame}

\begin{frame}{Softmax vs ArcFace}
    \textbf{Limitations of Standard Softmax}
    \begin{itemize}
        \item Huge number of classes ($10^5+$) $\rightarrow$ computationally expensive.
        \item Open-set problem: New cards are released frequently.
    \end{itemize}
    
    \vspace{0.5cm}
    
    \textbf{Metric Learning Approach (ArcFace)}
    \begin{itemize}
        \item Learning a mapping to a hypersphere embedding space.
        \item Intra-class \textit{compactness}. Images of the same card are pulled together.
        \item Intra-class \textit{separability}. Different cards are pushed apart.
        \item Inference via \textit{Cosine Similarity} (Nearest Neighbor).
    \end{itemize}
\end{frame}

\begin{frame}{Geometric Interpretation}
    \centering
    \tikzset{
        >=Stealth,
        feature_point/.style={circle, inner sep=1.5pt, fill=black!70},
        class1_point/.style={feature_point, fill=academicBlue},
        class2_point/.style={feature_point, fill=academicRed},
        weight_vec/.style={->, thick, shorten >=1pt},
        boundary_line/.style={dashed, thick, black!60},
        margin_area/.style={fill=marginGray, opacity=0.5},
        axis_style/.style={->, thin, black!40}
    }
    \vspace{1em}
    \begin{columns}[b]
        \begin{column}{0.54\textwidth}
            \centering
            \begin{tikzpicture}[scale=2.5]
                \coordinate (O) at (0,0);
                \draw[axis_style] (-0.2,0) -- (1.8,0);
                \draw[axis_style] (0,-0.2) -- (0,1.8);
            
                \def\angWone{30}
                \def\angWtwo{130}
                \pgfmathsetmacro{\angBoundary}{(\angWone + \angWtwo)/2}

                \coordinate (W1_tip) at (\angWone:1.4);
                \coordinate (W2_tip) at (\angWtwo:1.2);

                \draw[weight_vec, academicBlue] (O) -- (W1_tip) node[right] {$\mathbf{W}_1$};
                \draw[weight_vec, academicRed] (O) -- (W2_tip) node[left] {$\mathbf{W}_2$};
                \draw[boundary_line] (O) -- (\angBoundary:1.8) node[above right, font=\footnotesize] {};

                \coordinate (X_sample) at (\angWone+15:1.1);
                \draw[->, thin, black!80] (O) -- (X_sample) node[midway, below, font=\footnotesize] {$\mathbf{x}_i$};
            
                \pic [draw, angle radius=0.5cm, "$\theta_{1,i}$", font=\footnotesize, angle eccentricity=1.3] {angle = X_sample--O--W1_tip};

                \foreach \r/\a in {1.0/20, 1.3/35, 0.8/25, 1.5/40, 1.1/15, 0.9/30, 1.2/10}
                    \node[class1_point] at (\a:\r) {};
                \foreach \r/\a in {1.1/120, 0.9/135, 1.4/125, 1.0/140, 1.2/115, 0.8/130, 1.3/145}
                    \node[class2_point] at (\a:\r) {};
            \end{tikzpicture}
            
            \vspace{0.2cm}
            \footnotesize{(a) Standard Softmax Loss}
        \end{column}

        \begin{column}{0.46\textwidth}
            \centering
            \begin{tikzpicture}[scale=1.7]
                \coordinate (O) at (0,0);
                \def\radius{1.5}
                \draw[thick, black!80] (O) circle (\radius);
                \node at (115:\radius+0.3) [font=\footnotesize] {Hypersphere};

                \def\angWone{30}
                \def\angWtwo{130}
                \def\marginDeg{25}
                \pgfmathsetmacro{\angBoundary}{(\angWone + \angWtwo)/2}
                \pgfmathsetmacro{\angMarginOne}{\angBoundary - \marginDeg}
                \pgfmathsetmacro{\angMarginTwo}{\angBoundary + \marginDeg}

                \coordinate (W1_pt) at (\angWone:\radius);
                \coordinate (W2_pt) at (\angWtwo:\radius+0.05);
                \coordinate (M1_pt) at (\angMarginOne:\radius);
                \coordinate (M2_pt) at (\angMarginTwo:\radius);

                \begin{scope}[on background layer]
                    \fill[marginGray] (O) -- (M1_pt) arc (\angMarginOne:\angMarginTwo:\radius) -- cycle;
                \end{scope}
            
                \draw[boundary_line, thin] (O) -- (M1_pt);
                \draw[boundary_line, thin] (O) -- (M2_pt);
                \draw[dashed, thin, black!30] (O) -- (\angBoundary:\radius);

                \draw[weight_vec, academicBlue] (O) -- (W1_pt) node[right] {$\mathbf{W}_1$};
                \draw[weight_vec, academicRed] (O) -- (W2_pt) node[left] {$\mathbf{W}_2$};

                \pic [draw, <->, angle radius=0.8*\radius cm, "$m$", font=\small, angle eccentricity=1.15, academicBlue] {angle = W1_pt--O--M1_pt};
                \pic [draw, <->, angle radius=0.8*\radius cm, "$m$", font=\small, angle eccentricity=1.15, academicRed] {angle = M2_pt--O--W2_pt};

                \foreach \a in {28, 32, 25, 35, 30, 22, 38}
                    \node[class1_point] at (\a:\radius) {};
                \foreach \a in {128, 132, 125, 135, 130, 122, 138}
                    \node[class2_point] at (\a:\radius) {};

                \node at (\angBoundary:0.6*\radius) [font=\footnotesize, align=center, rotate=\angBoundary-90] {Margin\\Gap};
            \end{tikzpicture}
            
            \vspace{0.2cm}
            \footnotesize{(b) ArcFace Loss}
        \end{column}
    \end{columns}
\end{frame}

\begin{frame}{Recognition Pipeline}
    \begin{figure}
        \includegraphics[scale=0.85]{./media/flow.jpg}
    \end{figure}
\end{frame}

\begin{frame}{Conclusion and Future Work}
    \textbf{Summary}
    \begin{itemize}
        \item Basic HW control operational.
        \item Recognition pipeline implemented (Synthetic Data + ArcFace).
        \item Support for MultiGPU training.
    \end{itemize}
    
    \vspace{0.5cm}
    
    \textbf{Next Steps}
    \begin{itemize}
        \item Hyperparameter sweep (\textit{Weights \& Biases}).
        \item Finalize mechanical construction.
        \item Full integration of the recognition module with the robotic control.
        \item Final evaluation on a physical test set.
    \end{itemize}
\end{frame}

\end{document}

